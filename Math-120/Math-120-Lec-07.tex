\documentclass[]{article}
\usepackage{lmodern}
\usepackage{amssymb,amsmath}
\usepackage{ifxetex,ifluatex}
\usepackage{fixltx2e} % provides \textsubscript
\ifnum 0\ifxetex 1\fi\ifluatex 1\fi=0 % if pdftex
  \usepackage[T1]{fontenc}
  \usepackage[utf8]{inputenc}
\else % if luatex or xelatex
  \ifxetex
    \usepackage{mathspec}
  \else
    \usepackage{fontspec}
  \fi
  \defaultfontfeatures{Ligatures=TeX,Scale=MatchLowercase}
\fi
% use upquote if available, for straight quotes in verbatim environments
\IfFileExists{upquote.sty}{\usepackage{upquote}}{}
% use microtype if available
\IfFileExists{microtype.sty}{%
\usepackage{microtype}
\UseMicrotypeSet[protrusion]{basicmath} % disable protrusion for tt fonts
}{}
\usepackage[margin=1in]{geometry}
\usepackage{hyperref}
\hypersetup{unicode=true,
            pdftitle={Math 120 Week 05},
            pdfauthor={Prof Miller},
            pdfborder={0 0 0},
            breaklinks=true}
\urlstyle{same}  % don't use monospace font for urls
\usepackage{color}
\usepackage{fancyvrb}
\newcommand{\VerbBar}{|}
\newcommand{\VERB}{\Verb[commandchars=\\\{\}]}
\DefineVerbatimEnvironment{Highlighting}{Verbatim}{commandchars=\\\{\}}
% Add ',fontsize=\small' for more characters per line
\usepackage{framed}
\definecolor{shadecolor}{RGB}{248,248,248}
\newenvironment{Shaded}{\begin{snugshade}}{\end{snugshade}}
\newcommand{\AlertTok}[1]{\textcolor[rgb]{0.94,0.16,0.16}{#1}}
\newcommand{\AnnotationTok}[1]{\textcolor[rgb]{0.56,0.35,0.01}{\textbf{\textit{#1}}}}
\newcommand{\AttributeTok}[1]{\textcolor[rgb]{0.77,0.63,0.00}{#1}}
\newcommand{\BaseNTok}[1]{\textcolor[rgb]{0.00,0.00,0.81}{#1}}
\newcommand{\BuiltInTok}[1]{#1}
\newcommand{\CharTok}[1]{\textcolor[rgb]{0.31,0.60,0.02}{#1}}
\newcommand{\CommentTok}[1]{\textcolor[rgb]{0.56,0.35,0.01}{\textit{#1}}}
\newcommand{\CommentVarTok}[1]{\textcolor[rgb]{0.56,0.35,0.01}{\textbf{\textit{#1}}}}
\newcommand{\ConstantTok}[1]{\textcolor[rgb]{0.00,0.00,0.00}{#1}}
\newcommand{\ControlFlowTok}[1]{\textcolor[rgb]{0.13,0.29,0.53}{\textbf{#1}}}
\newcommand{\DataTypeTok}[1]{\textcolor[rgb]{0.13,0.29,0.53}{#1}}
\newcommand{\DecValTok}[1]{\textcolor[rgb]{0.00,0.00,0.81}{#1}}
\newcommand{\DocumentationTok}[1]{\textcolor[rgb]{0.56,0.35,0.01}{\textbf{\textit{#1}}}}
\newcommand{\ErrorTok}[1]{\textcolor[rgb]{0.64,0.00,0.00}{\textbf{#1}}}
\newcommand{\ExtensionTok}[1]{#1}
\newcommand{\FloatTok}[1]{\textcolor[rgb]{0.00,0.00,0.81}{#1}}
\newcommand{\FunctionTok}[1]{\textcolor[rgb]{0.00,0.00,0.00}{#1}}
\newcommand{\ImportTok}[1]{#1}
\newcommand{\InformationTok}[1]{\textcolor[rgb]{0.56,0.35,0.01}{\textbf{\textit{#1}}}}
\newcommand{\KeywordTok}[1]{\textcolor[rgb]{0.13,0.29,0.53}{\textbf{#1}}}
\newcommand{\NormalTok}[1]{#1}
\newcommand{\OperatorTok}[1]{\textcolor[rgb]{0.81,0.36,0.00}{\textbf{#1}}}
\newcommand{\OtherTok}[1]{\textcolor[rgb]{0.56,0.35,0.01}{#1}}
\newcommand{\PreprocessorTok}[1]{\textcolor[rgb]{0.56,0.35,0.01}{\textit{#1}}}
\newcommand{\RegionMarkerTok}[1]{#1}
\newcommand{\SpecialCharTok}[1]{\textcolor[rgb]{0.00,0.00,0.00}{#1}}
\newcommand{\SpecialStringTok}[1]{\textcolor[rgb]{0.31,0.60,0.02}{#1}}
\newcommand{\StringTok}[1]{\textcolor[rgb]{0.31,0.60,0.02}{#1}}
\newcommand{\VariableTok}[1]{\textcolor[rgb]{0.00,0.00,0.00}{#1}}
\newcommand{\VerbatimStringTok}[1]{\textcolor[rgb]{0.31,0.60,0.02}{#1}}
\newcommand{\WarningTok}[1]{\textcolor[rgb]{0.56,0.35,0.01}{\textbf{\textit{#1}}}}
\usepackage{graphicx,grffile}
\makeatletter
\def\maxwidth{\ifdim\Gin@nat@width>\linewidth\linewidth\else\Gin@nat@width\fi}
\def\maxheight{\ifdim\Gin@nat@height>\textheight\textheight\else\Gin@nat@height\fi}
\makeatother
% Scale images if necessary, so that they will not overflow the page
% margins by default, and it is still possible to overwrite the defaults
% using explicit options in \includegraphics[width, height, ...]{}
\setkeys{Gin}{width=\maxwidth,height=\maxheight,keepaspectratio}
\IfFileExists{parskip.sty}{%
\usepackage{parskip}
}{% else
\setlength{\parindent}{0pt}
\setlength{\parskip}{6pt plus 2pt minus 1pt}
}
\setlength{\emergencystretch}{3em}  % prevent overfull lines
\providecommand{\tightlist}{%
  \setlength{\itemsep}{0pt}\setlength{\parskip}{0pt}}
\setcounter{secnumdepth}{0}
% Redefines (sub)paragraphs to behave more like sections
\ifx\paragraph\undefined\else
\let\oldparagraph\paragraph
\renewcommand{\paragraph}[1]{\oldparagraph{#1}\mbox{}}
\fi
\ifx\subparagraph\undefined\else
\let\oldsubparagraph\subparagraph
\renewcommand{\subparagraph}[1]{\oldsubparagraph{#1}\mbox{}}
\fi

%%% Use protect on footnotes to avoid problems with footnotes in titles
\let\rmarkdownfootnote\footnote%
\def\footnote{\protect\rmarkdownfootnote}

%%% Change title format to be more compact
\usepackage{titling}

% Create subtitle command for use in maketitle
\providecommand{\subtitle}[1]{
  \posttitle{
    \begin{center}\large#1\end{center}
    }
}

\setlength{\droptitle}{-2em}

  \title{Math 120 Week 05}
    \pretitle{\vspace{\droptitle}\centering\huge}
  \posttitle{\par}
    \author{Prof Miller}
    \preauthor{\centering\large\emph}
  \postauthor{\par}
      \predate{\centering\large\emph}
  \postdate{\par}
    \date{03 March, 2020}


\begin{document}
\maketitle

{
\setcounter{tocdepth}{3}
\tableofcontents
}
\hypertarget{friday-after-the-exam}{%
\section{Friday (After the Exam)}\label{friday-after-the-exam}}

\hypertarget{continuous-distros}{%
\subsection{Continuous distros}\label{continuous-distros}}

Continuous distros are kinda weird!

Let's do an easy example: uniform distro. I.e., the ``flat line''
distro.

Example: the amount of time that students spend on an exam follows a
uniform distribution that ranges between 40min and 60min.

Note: this is continuous, since time take could be any real number (not
just discrete values). Ex: could take 48.2146324 min.

For all continuous distros, probability is found by calculating area
under the ``curve''.\\
Important fact: since total probability always 100\%, total area is
always 1. SO, if your uniform distro ranges from a to b, the height is
always 1/(b-a).

Finding probabilities for the unif is easy: just rectangles!

Ex 1: with the exam distro above, what's the probability that a student
spends between 45 and 55 minutes on the exam?

Answer: Area of rectangle! The base is 10, the height is 1/20:

\begin{Shaded}
\begin{Highlighting}[]
\DecValTok{10}\OperatorTok{*}\DecValTok{1}\OperatorTok{/}\DecValTok{20}
\end{Highlighting}
\end{Shaded}

\begin{verbatim}
## [1] 0.5
\end{verbatim}

There's a 50\% chance a student spends between 45 and 55 min.

Ex 2: What's the probability that a student spends more than 53 minutes
on the exam?

\begin{Shaded}
\begin{Highlighting}[]
\DecValTok{1}\OperatorTok{/}\DecValTok{20}\OperatorTok{*}\DecValTok{7}
\end{Highlighting}
\end{Shaded}

\begin{verbatim}
## [1] 0.35
\end{verbatim}

A: there's a 35\% chance.

Ex 3: Same exam distro. What's the probability that student spends
exactly 50.0000000000000\ldots{} minutes on the exam?

A: zero! woooah! weird dude!

Ex 3: Same example. WHat's the prob that the student spends exactly
47.0000000\ldots{} min?

Still zero!

Moral of the story:

\begin{itemize}
\tightlist
\item
  If x is continuous, the probability that X=any \# is always zero!!! If
  width = 0, no area, so no probability!
\item
  For continuous distros, it only makes sense to think about ranges of
  values for X
\end{itemize}

Ex: the mean height for women the us is 64" tall. What's the probability
that a random woman EXACTLY 64.00000000" tall?

Zero!

Ex) Exam distro. Which is bigger:

P(X\textgreater{}50)

or

P(X\textgreater{}=50)

?

A: they're the same! P(X=50)=0, doesn't change the
area/region/probability!

Moral of the story: for continuous distros,

P(X \textless{}= blah) == P(X\textless{} blah)

same for \textgreater{}, etc etc. I.e., doesn't matter if we include the
endpoint. Big difference from discrete distros!

\hypertarget{normal-distro}{%
\subsection{Normal distro}\label{normal-distro}}

The normal distro (bell curve distribution) is also a continuous distro,
and super super common in real life.

You already know basics of normal probabilities from the empirical rule!
BUT, we're going to expand that to include ANY z-score, not just +-1,
+-2, +-3.

Normal distros work just like any continuous distro: probability = area
under the curve (picture on board/notes)

Bad news: the function for the normal distribution is nasty, and there's
NO WAY to find areas by hand (outside of the empirical rule). We always
need help. Two ways:

\begin{enumerate}
\def\labelenumi{\arabic{enumi})}
\tightlist
\item
  Table! I.e., a piece of paper with normal probabilities on it!
\item
  Use software (R, excel, Minitab, etc etc)
\end{enumerate}

\hypertarget{using-the-z-table}{%
\subsection{Using the Z table}\label{using-the-z-table}}

The ``Z'' distro has a special meaning: it's the normal distro with mean
= 0, std dev =1.

I.e., it's the z-score distribution!

Examples. Compute:

\begin{enumerate}
\def\labelenumi{\arabic{enumi})}
\tightlist
\item
  P(Z \textless{} -1.23) = .1093
\end{enumerate}

Important note: the areas on the Z tables are always ``to the
left''/``below''

\begin{enumerate}
\def\labelenumi{\arabic{enumi})}
\setcounter{enumi}{1}
\tightlist
\item
  P(Z \textgreater{} -1.23) = 1 - P(Z\textless{} - 1.23) = 1-.1093
\end{enumerate}

\begin{Shaded}
\begin{Highlighting}[]
\DecValTok{1}\FloatTok{-.1093}
\end{Highlighting}
\end{Shaded}

\begin{verbatim}
## [1] 0.8907
\end{verbatim}

\begin{enumerate}
\def\labelenumi{\arabic{enumi})}
\setcounter{enumi}{2}
\tightlist
\item
  P( -1.05 \textless{} Z \textless{} -0.03 )
\end{enumerate}

\hypertarget{friday}{%
\section{Friday}\label{friday}}

\hypertarget{continuous-distros-1}{%
\subsection{Continuous Distros}\label{continuous-distros-1}}

Continuous are weird!

First, let's think about the ``uniform'' distribution (i.e.~the ``flat
line'' distro). It's the easiest continuous distro, but teaces all the
important concepts.

\hypertarget{example-exam-times}{%
\subsection{Example: Exam times}\label{example-exam-times}}

Suppose that X=the amount of time that student spend on an exam, and
that X has a uniform distribution that ranges from 40min to 60min.
Picture on board.

For all continuous distros (including unif), probability correponds to
the area under the ``curve''.

Important property: since total probability is always 100\%, that means
total area under probability curve is alway 1.

Because our uniform dist is a rectangle, we always know it's height. If
it ranges from a to b (here, a=40, b=60), then the height of the
line/curve is always 1/(b-a).

Compute the following:

\begin{enumerate}
\def\labelenumi{\arabic{enumi})}
\tightlist
\item
  P( 45 \textless{} X \textless{} 55) = 1/20*10
\end{enumerate}

\begin{Shaded}
\begin{Highlighting}[]
\DecValTok{1}\OperatorTok{/}\DecValTok{20}\OperatorTok{*}\DecValTok{10}
\end{Highlighting}
\end{Shaded}

\begin{verbatim}
## [1] 0.5
\end{verbatim}

There's a 50\% chance the student takes that long.

\begin{enumerate}
\def\labelenumi{\arabic{enumi})}
\setcounter{enumi}{1}
\tightlist
\item
  What's the prob that a student takes more than 53 min?
\end{enumerate}

\begin{Shaded}
\begin{Highlighting}[]
\DecValTok{7}\OperatorTok{*}\DecValTok{1}\OperatorTok{/}\DecValTok{20}
\end{Highlighting}
\end{Shaded}

\begin{verbatim}
## [1] 0.35
\end{verbatim}

\begin{enumerate}
\def\labelenumi{\arabic{enumi})}
\setcounter{enumi}{2}
\tightlist
\item
  What's the prob that a student takes EXACTLY 50.000000000\ldots{} min?
\end{enumerate}

A: zero! Woah, weird!!!

\begin{enumerate}
\def\labelenumi{\arabic{enumi})}
\setcounter{enumi}{3}
\tightlist
\item
  P(X == 47.0000\ldots{}..)
\end{enumerate}

A: ALso zero! Woah!

In fact, for ANY contintinuous distribution:

\begin{itemize}
\tightlist
\item
  P(X == any number) = 0 always. ANy continuous X, any number.\\
\item
  For continuous variables, it only makes sense to talk about RANGES of
  values for X.
\end{itemize}

\begin{enumerate}
\def\labelenumi{\arabic{enumi})}
\setcounter{enumi}{4}
\tightlist
\item
  Same exam distro. Which is bigger:
\end{enumerate}

P(X \textgreater{} 50)

or

P(X \textgreater{}= 50)

?

THEY'RE EQUAL!

Moral: for continuous distros, doesn't matter if we include the
endpoint.

P(X\textless{} 50) == P(X \textless{}= 50)

This is in contrast to discrete distros like binomial, geometric,
poisson. For those, it matters a bunch if you include the endpoint!

\hypertarget{normal-distro-1}{%
\subsection{Normal Distro}\label{normal-distro-1}}

The normal distro (ie the bell curve) is the most useful continuous
distribution. Pops up everywhere!

Recall, you already know the empirical rule, so you already know the
basics of normal probabilities. BUT, only works for z = 0, +-1, +-2,
+-3.

Normal dist works just like any other continuous distro: it has a
function (pdf, probability density function), and we find probabilities
by calculating area under the function.

Bad news: the function (pdf) for the normal distro is nasty. It's
impossible to calculate areas by hand. We always need help! Two main
ways:

\begin{enumerate}
\def\labelenumi{\arabic{enumi})}
\tightlist
\item
  Probability tables. I.e., dead tree. Z table.
\item
  Software. (R, Excel, Minitab, SASS)
\end{enumerate}

\hypertarget{using-the-z-table-1}{%
\subsection{Using the z table}\label{using-the-z-table-1}}

WARNING: There are several different ways to construct a z table. Use
our book!

Q: Why the ``Z'' table?

A: There are lots of different normal distribution with different center
and spread.

The Z distro is the ``STANDARD NORMAL'' distro: mean = 0, std dev = 1.

The table always shows area to the LEFT.

Examples. Compute:

\begin{enumerate}
\def\labelenumi{\arabic{enumi})}
\item
  P(Z \textless{} -1.23) = 0.1093. There's a 10.93\% chance that
  Z\textless{}-1.23
\item
  P(Z \textgreater{} -1.23) = 1 - 0.1093 = .8907
\end{enumerate}

\begin{Shaded}
\begin{Highlighting}[]
\DecValTok{1}\FloatTok{-0.1093}
\end{Highlighting}
\end{Shaded}

\begin{verbatim}
## [1] 0.8907
\end{verbatim}

\begin{enumerate}
\def\labelenumi{\arabic{enumi})}
\setcounter{enumi}{2}
\tightlist
\item
  P( -1.74 \textless{} Z \textless{} -0.19 ) = P(Z \textless{} -.19) -
  P(Z \textless{} -1.74) = .4247 - .0409
\end{enumerate}

\begin{Shaded}
\begin{Highlighting}[]
\FloatTok{.4247} \OperatorTok{-}\StringTok{ }\FloatTok{.0409}
\end{Highlighting}
\end{Shaded}

\begin{verbatim}
## [1] 0.3838
\end{verbatim}

\hypertarget{monday}{%
\section{Monday}\label{monday}}

\hypertarget{warm-ups}{%
\subsection{Warm-ups}\label{warm-ups}}

\begin{enumerate}
\def\labelenumi{\arabic{enumi})}
\tightlist
\item
  At a bus stop, the amount of time that travellers must wait ranges
  uniformly between 5 and 15 minutes.
\end{enumerate}

\begin{enumerate}
\def\labelenumi{\alph{enumi})}
\tightlist
\item
  What's the probability that a traveller must wait less than (or equal
  to) 8 min?
\end{enumerate}

\begin{Shaded}
\begin{Highlighting}[]
\DecValTok{3}\OperatorTok{*}\DecValTok{1}\OperatorTok{/}\DecValTok{10}
\end{Highlighting}
\end{Shaded}

\begin{verbatim}
## [1] 0.3
\end{verbatim}

\begin{enumerate}
\def\labelenumi{\alph{enumi})}
\setcounter{enumi}{1}
\tightlist
\item
  What's the probability that a traveller must wait between 10 and 13
  min?
\end{enumerate}

Same area, same probability! 30\%.

\begin{enumerate}
\def\labelenumi{\alph{enumi})}
\setcounter{enumi}{2}
\tightlist
\item
  What's the prob that they wait exactly 11 min?
\end{enumerate}

Cts, so prob = 0.

\hypertarget{z-table-practice}{%
\subsection{Z table practice}\label{z-table-practice}}

Suppose Z has a std normal dist (note: z always means standard normal,
mean=0 and std dev =1). Compute:

\begin{enumerate}
\def\labelenumi{\alph{enumi})}
\item
  P(Z \textless{} 1.52) = .9357
\item
  P(Z \textgreater{} -0.44) = 1-.3300 = 0.6700
\item
  P( -0.19 \textless{} Z \textless{} 3.01) = P(Z \textless{} 3.01) - P(Z
  \textless{} -0.19) = .9987 - .4247
\end{enumerate}

\begin{Shaded}
\begin{Highlighting}[]
\FloatTok{.9987-.4247}
\end{Highlighting}
\end{Shaded}

\begin{verbatim}
## [1] 0.574
\end{verbatim}

\begin{enumerate}
\def\labelenumi{\alph{enumi})}
\setcounter{enumi}{3}
\item
  P(Z \textless{} -4.3) = approx 0!
\item
  P(Z \textless{} 5.1) = approx 1!
\end{enumerate}

\hypertarget{z-table-backwards}{%
\subsection{Z table backwards}\label{z-table-backwards}}

So far: I give you Z, you tell me area.

BUT, what if we know an area/probability and we want correponding
z-score cutoff?

Ex: How big must Z be in order to be in the top 10\% of the distro?
(i.e., the 90th percentile)

Idea: find the area (as close as possible) to 90\%. Remember, table
always shows left area!

Closest area: .8997, z score is 1.28.

Example: How big must a z score be in order to be in the top 25\% (i.e.,
75th percentile) ?

z = 0.67

Forwards: start with z score, find area. Backwards: start with
area/probability, find z score

\hypertarget{special-type-middle}\label{special-type-middle}}

Ex: what are the z cutoff for the middle 50\% of the z distro?

A: Between z=-.67 and z=+.67 lies the middle 50\% of the z distro

Ex:what are the z cutoff for the middle 95\% of the z distro?

The middle 95\% of the data lies between z=-1.96 and z=+1.96

Note: this is a more precise version of the emprical rule!

Quiz on wed: quiz and z table basics (forwards and backwards)

\hypertarget{monday-1}{%
\section{Monday}\label{monday-1}}

\hypertarget{warm-up}{%
\subsection{Warm-up}\label{warm-up}}

Example) At a bus stop, commuters wait between 5 and 15 minutes,
following a uniform distribution. What't the probability\ldots{}

\begin{enumerate}
\def\labelenumi{\alph{enumi})}
\tightlist
\item
  A commuter must wait less than (or equal to) 8 min?
\end{enumerate}

\begin{Shaded}
\begin{Highlighting}[]
\DecValTok{3}\OperatorTok{*}\DecValTok{1}\OperatorTok{/}\DecValTok{10}
\end{Highlighting}
\end{Shaded}

\begin{verbatim}
## [1] 0.3
\end{verbatim}

\begin{enumerate}
\def\labelenumi{\alph{enumi})}
\setcounter{enumi}{1}
\tightlist
\item
  Must wait between 10 and 13 min?
\end{enumerate}

Same! Still a rectangle of width three, same region, same area, same
prob!

\begin{enumerate}
\def\labelenumi{\alph{enumi})}
\setcounter{enumi}{2}
\tightlist
\item
  Exactly 11 min?
\end{enumerate}

0! FOr any cts dist, P(X=exactly some number) = 0

\hypertarget{z-table-practice-1}{%
\subsection{Z table practice}\label{z-table-practice-1}}

Suppose that Z has a standard normal dist (Note: in stats, z ALWAYS
means standard normal distribution, mean of zero and std dev of 1),
compute:

\begin{enumerate}
\def\labelenumi{\alph{enumi})}
\item
  P(Z \textless{} 1.83) = .9664
\item
  P(Z \textgreater{} -0.41) = 1 - .3409 = .6591
\end{enumerate}

\begin{Shaded}
\begin{Highlighting}[]
\DecValTok{1}\FloatTok{-.3409}
\end{Highlighting}
\end{Shaded}

\begin{verbatim}
## [1] 0.6591
\end{verbatim}

\begin{enumerate}
\def\labelenumi{\alph{enumi})}
\setcounter{enumi}{2}
\tightlist
\item
  P( -0.25 \textless{} Z \textless{} 3.12) = P(Z \textless{} 3.12) - P(Z
  \textless{} -0.25) = .5978
\end{enumerate}

\begin{Shaded}
\begin{Highlighting}[]
\FloatTok{.9991} \OperatorTok{-}\StringTok{ }\FloatTok{.4013}
\end{Highlighting}
\end{Shaded}

\begin{verbatim}
## [1] 0.5978
\end{verbatim}

\begin{enumerate}
\def\labelenumi{\alph{enumi})}
\setcounter{enumi}{3}
\item
  P(Z \textless{} -4.21) \textasciitilde{} 0 (\textasciitilde{} =
  approx)
\item
  P(Z \textless{} 5.30) \textasciitilde{} 1
\end{enumerate}

\hypertarget{z-table-backwards-1}{%
\subsection{Z table: backwards}\label{z-table-backwards-1}}

So far: I tell you Z, you tell me prob/area

Now, backwards: I tell you area/prob, you tell me z score cutoff

Example) How large must a z score be in order to be in the top 10\% of
the distro (90th percentile)?

Note: table always shows area to the left, so we use 90\% or 0.90.

Idea: Find the closest area to 0.90, then get z score

Closest area: .8997 -\textgreater{} z = 1.28

Example) How large must a z score be to be in the top 20\% of the data
(i.e.~80th percentile).

Closest area: 0.7995. Z score = 0.84

\hypertarget{middle}\label{middle}}

Example: find the zscores that mark off the middle 60\% of the data.

Between z=-.84 and z=+.84 is the middle 60\% of the data.

Example: find the z scores that mark off the middle 95\% of the data.

The middle 95\% of the data lies between z=-1.96 and z=+1.96

Quiz on wed: uniform dist, basic z-table stuff

Look at 4.1 exercise

\hypertarget{tuesday}{%
\section{Tuesday}\label{tuesday}}

\hypertarget{using-the-z-table-in-practical-real-world-scenarios}{%
\subsection{Using the z table in practical real-world
scenarios}\label{using-the-z-table-in-practical-real-world-scenarios}}

In the real world, there are lots of different normal dists.
SPecifically, different centers and spread! Example: human height.

To use the z table (std normal, mean=0, stdev = 1) in these scenarios,
we need to do a little extra work.

\hypertarget{forwards}{%
\subsection{Forwards}\label{forwards}}

Remember: forwards means we start with a value, need to find a
probability/area.

For real-world data, we need to find z scores on our own! Exactly the
same as before.

Formula:

z = (x-mu)/sigma = (x-xbar)/s

So, first get z, then use table. Bam!

Example) Height for adult men (20-29) in the US follows a normal dist
with mean = 69" and std dev = 2.7".

What's the prob that a randomly selected man is less than 6' tall? (72")

First: z score. x = 72. mu = 69, sigma = 2.7

\begin{Shaded}
\begin{Highlighting}[]
\NormalTok{(}\DecValTok{72-69}\NormalTok{)}\OperatorTok{/}\FloatTok{2.7}
\end{Highlighting}
\end{Shaded}

\begin{verbatim}
## [1] 1.111111
\end{verbatim}

So, P(X\textless{}72) = P(Z \textless{} 1.11) = .8665

There's about an 86.65\% chance that a rando man is less than 6' tall.

Example) Height for adult women follows a normal dist with mean = 64"
and std dev = 2.4``. What's the probility that a rando woman is between
5'6'' (66``) and 5'8'' (68").

z-scores:

\begin{Shaded}
\begin{Highlighting}[]
\NormalTok{(}\DecValTok{68-64}\NormalTok{)}\OperatorTok{/}\FloatTok{2.4}
\end{Highlighting}
\end{Shaded}

\begin{verbatim}
## [1] 1.666667
\end{verbatim}

\begin{Shaded}
\begin{Highlighting}[]
\NormalTok{(}\DecValTok{66-64}\NormalTok{)}\OperatorTok{/}\FloatTok{2.4}
\end{Highlighting}
\end{Shaded}

\begin{verbatim}
## [1] 0.8333333
\end{verbatim}

P( 0.83 \textless{} Z \textless{} 1.67) = P(Z\textless{}1.67) - P(Z
\textless{} 0.83)

\begin{Shaded}
\begin{Highlighting}[]
\FloatTok{.9525-.7967}
\end{Highlighting}
\end{Shaded}

\begin{verbatim}
## [1] 0.1558
\end{verbatim}

There's a 15.58\% chance that rando woman is between 5'6" and 5'8".

\hypertarget{backwards}{%
\subsection{Backwards}\label{backwards}}

Remember: backwards means we start with prob/area, find
cutoff/value/measurement/obs.

Example) How tall must a woman be in order to be in the top 5\% of
womens' heights?

First: look for area in table, find z-score.

Note: if two z-scores are equally good, just spilt them. Ie. average.

Here: z=1.64 and z=1.65 are equally close, use z=1.645.

Got z, what's x?

\begin{Shaded}
\begin{Highlighting}[]
\FloatTok{1.645}\OperatorTok{*}\FloatTok{2.4}\OperatorTok{+}\DecValTok{64}
\end{Highlighting}
\end{Shaded}

\begin{verbatim}
## [1] 67.948
\end{verbatim}

She'd have to be 5'7.948``. I.e., 67.948''.

Note, since z = (x-mu)/sigma,

x = z*sigma+mu

Example) How tall must a rando man be in order to be in the top 15\% of
mens' heights? Recall, mean = 69, sigma = 2.7.

\begin{Shaded}
\begin{Highlighting}[]
\FloatTok{1.04}\OperatorTok{*}\FloatTok{2.7}\OperatorTok{+}\DecValTok{69}
\end{Highlighting}
\end{Shaded}

\begin{verbatim}
## [1] 71.808
\end{verbatim}

He'd have to be 71.8" tall in order to be in the top 15\%.

\hypertarget{sampling-distro}{%
\subsection{Sampling distro}\label{sampling-distro}}

Suppose I roll a dice a bunch of times, and make a histogram of the
resulsts.

What shape would the histo have? Expect a flat line.

Dice rolling activity

Take turns rolling the dice. Every student rolls 50 times. While one
student rolls, the other records.

\hypertarget{tues}{%
\subsection{Tues}\label{tues}}

\hypertarget{z-table-for-real-world-scenarios}{%
\subsection{Z table for real-world
scenarios}\label{z-table-for-real-world-scenarios}}

There are lots of normal dists out there (different centers and
spreads), but the z table only shows STANDARD normal (mean=0, stdev=1).

To use z table, we first need to convert to standard normal.

This is easy: z-scores!

\hypertarget{forwards-problems}{%
\subsection{Forwards problems}\label{forwards-problems}}

Recall, forwards: start with a value/observation, find prob/area.

For real-world data, same thing, just find z-scores first!

Example) Height for men in the US (20-29) follows a normal dist with
mean = 69" and stdev = 2.7``. What's the prob that a randomly selected
man is less than 6' tall (72'')?

X = 72, mu = 69, sigma = 2.7

First, z score:

\begin{Shaded}
\begin{Highlighting}[]
\NormalTok{(}\DecValTok{72-69}\NormalTok{)}\OperatorTok{/}\FloatTok{2.7}
\end{Highlighting}
\end{Shaded}

\begin{verbatim}
## [1] 1.111111
\end{verbatim}

P(X\textless{}72) = P(Z\textless{}1.11) = 0.8665

There's about an 86.65\% chance a rando man is less than 72".

Example) Height for women in the US follows a normal dist with mean 64"
and stddev = 2.4``. What's the prob that a rando woman is between 5'6''
(66``) and 5'8'' (68")?

\begin{Shaded}
\begin{Highlighting}[]
\NormalTok{(}\DecValTok{68-64}\NormalTok{)}\OperatorTok{/}\FloatTok{2.4}
\end{Highlighting}
\end{Shaded}

\begin{verbatim}
## [1] 1.666667
\end{verbatim}

\begin{Shaded}
\begin{Highlighting}[]
\NormalTok{(}\DecValTok{66-64}\NormalTok{)}\OperatorTok{/}\FloatTok{2.4}
\end{Highlighting}
\end{Shaded}

\begin{verbatim}
## [1] 0.8333333
\end{verbatim}

P(66 \textless{} X \textless{} 68) = P( 0.83 \textless{} Z \textless{}
1.67) = P(Z\textless{}1.67) - P(Z\textless{}0.83) =

\begin{Shaded}
\begin{Highlighting}[]
\FloatTok{.9525-.7967}
\end{Highlighting}
\end{Shaded}

\begin{verbatim}
## [1] 0.1558
\end{verbatim}

There's about a 15.58\% chance that rando woman is between 5'6" and
5'8".

\hypertarget{backwards-1}{%
\subsection{Backwards}\label{backwards-1}}

Backwards: start with prob/area, find a measurement/observation/cutoff/x

Example: IQ follows a normal dist with mean 100 and std dev 15. How high
must one's IQ be in order to be in the top 5\% of IQs?

Note: if two z-scores are equally good, split the difference. Ie
average.

Here, z=1.64 and z=1.65 are equally good, use z = 1.645.

THus, X:

\begin{Shaded}
\begin{Highlighting}[]
\FloatTok{1.645}\OperatorTok{*}\DecValTok{15}\OperatorTok{+}\DecValTok{100}
\end{Highlighting}
\end{Shaded}

\begin{verbatim}
## [1] 124.675
\end{verbatim}

One's IQ would have to be at least 124.675 in order to make it to the
top 5\%.

Here, we plugged into zscore formula.

Solve backwards:

x = z*sigma + mu

Example) Men's heights. Mean = 69. Sigma = 2.7. How tall must rando man
be in order to be in the top 15\% of men's heights?

\begin{Shaded}
\begin{Highlighting}[]
\FloatTok{1.04}\OperatorTok{*}\FloatTok{2.7}\OperatorTok{+}\DecValTok{69}
\end{Highlighting}
\end{Shaded}

\begin{verbatim}
## [1] 71.808
\end{verbatim}

He'd have to be at least 71.808" tall to be in the top 15\% of heights.


\end{document}
